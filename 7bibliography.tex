
\begin{thebibliography}{99}
\addcontentsline{toc}{chapter}{Bibliography}


\bibitem{socialite} J. Seo, S. Guo, M. S. Lam: \textit{SociaLite: Datalog extensions for efficient social network analysis}, ICDE 2013.

\bibitem{distsoc} J. Seo, J. Park, J. Shin, M. S. Lam: \textit{Distributed SociaLite: A Datalog-Based Language for Large-Scale Graph Analysis}. PVLDB 6(14): 1906-1917, 2013.

\bibitem{fod} S. Abiteboul, R. Hull, and V. Vianu: \textit{Foundations of Databases}. Addison-Wesley. 1995.

\bibitem{wfom} T.J. Ameloot, B. Ketsman, F. Neven, D. Zinn: \textit{Weaker Forms of Monotonicity for Declarative Networking: a more fine-grained answer to the CALM-conjecture}, PODS 2014.

\bibitem{pagerank} S. Brin, L. Page: \textit{The anatomy of a large-scale hypertextual web search engine.} In WWW’98, 1998.

\bibitem{mapreduce} J. Dean, S. Ghemawat: \textit{MapReduce: simplified data processing on large clusters}, Proceedings of the 6th conference on Symposium on Opearting Systems Design \& Implementation, 2004.

\bibitem{pregel} G. Malewicz, M. H. Austern, A. J.C. Bik, J. C. Dehnert, I. Horn, N. Leiser, G. Czajkowski: \textit{Pregel: a system for large-scale graph processing}, Proceedings of the 2010 ACM SIGMOD International Conference on Management of data, 2010.

\bibitem{giraphpp} Y. Tian, A. Balmin, S. A. Corsten, S. Tatikonda, J. McPherson: \textit{From "Think Like a Vertex" to "Think Like a Graph"}, Proceedings of the VLDB Endowment, 2013.

\bibitem{gps}S. Salihoglun, J. Widom: \textit{GPS: A Graph Processing System.} SSDBM, July 2013.

\bibitem{deltastep} U. Meyer, P. Sanders: \textit{Delta-stepping: A parallel single source shortest path algorithm.} ESA, 1998.

\bibitem{ullman} A. Rajaraman, J. D. Ullman: \textit{Mining of Massive Datasets}, Cambridge University Press, New York, NY, 2011.

\bibitem{coddrelmodel} E. F. Codd: \textit{A relational model of data for large shared data banks}, Communications of the ACM, v.13 n.6, 1970.

\bibitem{RBS87} R. Ramakrishnan, R. Bancilhon, A. Silberschatz:  \textit{Safety of recursive horn clauses with infinite relations}, Proc. ACM Symp. on Principles of Database Systems, 1987.
\bibitem{KRS88a} M. Kifer, R. Ramakrishnan, A. Silberschatz:  \textit{An axiomatic approach to deciding query safety in deductive databases}, Proc. ACM Symp. on Principles of Database Systems, 1988.
\bibitem{KRS88b} R. Krishnamurthy, R. Ramakrishnan, O. Shmueli:  \textit{A framework for testing safety and effective computability of extended Datalog}, Proc. ACM SIGMOD Symp. on the Management of Data, 1988.
\bibitem{SV89} Y. Sagiv, M. Y. Vardi:  \textit{Safety of datalog queries over infinite databases}.  Proc. ACM Symp. on Principles of Database Systems, 1989.

\bibitem{graphlabwww} http://graphlab.org, Accessed: September 18th, 2014.
\bibitem{graphlab2} Y. Low, J. Gonzalez, A. Kyrola, D. Bickson, C. Guestrin, J. M. Hellerstein: \textit{Distributed GraphLab: A Framework for Machine Learning and Data Mining in the Cloud}, PVLDB 2012.

\bibitem{graphlab} Y. Low, J. Gonzalez, A. Kyrola, D. Bickson, C. Guestrin, J. M. Hellerstein: \textit{GraphLab: A New Parallel Framework for Machine Learning.} Conference on Uncertainty in Artificial Intelligence, 2010.

\bibitem{spark} M. Zaharia, M. Chowdhury, M. J. Franklin, S. Shenker, I. Stoica: \textit{Spark: Cluster Computing with Working Sets}, HotCloud 2010.
\bibitem{spark2} M. Zaharia, M. Chowdhury, T. Das, A. Dave, J. Ma, M. McCauley, M. J. Franklin, S. Shenker, I. Stoica: \textit{Resilient distributed datasets: a fault-tolerant abstraction for in-memory cluster computing.} In Proceedings of the 9th USENIX conference on Networked Systems Design and Implementation (NSDI'12), 2012.
\bibitem{pointanalysis} J. Whaley, M. S. Lam: \textit{Cloning-based context-sensitive pointer alias analyses using binary decision diagrams} In PLDI, 2004.
\bibitem{boomanalysis} P. Alvaro, T. Condie, N. Conway, K. Elmeleegy, J. M. Hellerstein, R. Sears: \textit{Boom analytics: Exploring data-centric, declarative programming for the cloud}, In EuroSys, 2010.
\bibitem{dataloganalysis} P. Alvaro, W. R. Marczak, N. Conway, J. M. Hellerstein, D. Maier, R. Sears: \textit{Dedalus: Datalog in time and space.} In Datalog, 2010.

\bibitem{bsp} L. G. Valiant: \emph{A Bridging Model for Parallel Computation.} Comm. ACM 33(8), 1990.

\bibitem{bgl} J. G. Siek, L. Lee, A. Lumsdaine: \textit{The Boost Graph Library: User Guide and Reference Manual.} Addison Wesley, 2002.
\bibitem{parallelbgl} D. Gregor, A. Lumsdaine: \textit{The Parallel BGL: A Generic Library for Distributed Graph Computations.} Proc. of Parallel Object-Oriented Scientific Computing (POOSC), July 2005.
\bibitem{GraphBase} D. E. Knuth: \textit{Stanford GraphBase: A Platform for Combinatorial Computing.} ACM Press, 1994.

\bibitem{logicblox} http://www.logicblox.com/technology.html, Accessed: September 18th, 2014.
\bibitem{datomic} http://www.datomic.com, Accessed: September 18th, 2014.
\bibitem{giraph} http://giraph.apache.com, Accessed: September 18th, 2014.
\bibitem{sparkwww} http://spark.apache.com, Accessed: September 18th, 2014.
\bibitem{hadoop} http://hadoop.apache.com, Accessed: September 18th, 2014.
\bibitem{githubspark} https://github.com/apache/spark, Accessed: September 18th, 2014.
\bibitem{giraphfb} https://www.facebook.com/notes/facebook-engineering/scaling-apache-giraph-to-a-\\trillion-edges/10151617006153920, Accessed: September 18th, 2014.
\bibitem{sparktoplevel} https://blogs.apache.org/foundation/entry/the\_apache\_software\_foundation\_\\announces50, Accessed: September 18th, 2014.
\bibitem{sparkgrowingcommunity} http://databricks.com/blog/2013/10/27/the-growing-spark-community.html, Accessed: September 18th, 2014.
\bibitem{mesos} http://mesos.apache.org, Accessed: September 18th, 2014.
\bibitem{ec2} http://aws.amazon.com, Accessed: September 18th, 2014.
\bibitem{piglatin} http://pig.apache.org, Accessed: September 18th, 2014.
\bibitem{hive} http://hive.apache.org, Accessed: September 18th, 2014.

\bibitem{magicsetsexist} M. Alviano, N. Leone, M. Manna, G. Terracina, P. Veltri: \textit{Magic-Sets for Datalog with Existential Quantifiers.}, Datalog 2012.
\bibitem{disjunctivedatalog} M. Alviano, W. Faber, N. Leone, M. Manna: \textit{Disjunctive datalog with existential quantifiers: Semantics, decidability, and complexity issues.} TPLP 12(4-5): 701-718, 2012.
\bibitem{datalogrelaunched} F. Bry, T. Furche, C. Ley, B. Marnette, B. Linse, S. Schaffert: \textit{Datalog relaunched: simulation unification and value invention}, Proceedings of the First international conference on Datalog Reloaded, Oxford, UK, 2010.
\bibitem{magicsets} F. Bancilhon, D. Maier, Y. Sagiv, J. D. Ullman: \textit{Magic sets and other strange ways to implement logic programs}. In Proceedings of the fifth ACM SIGACT-SIGMOD symposium on Principles of database systems, PODS 1986.
\bibitem{subsumptivequeries} K. Tuncay Tekle, Y. A. Liu: \textit{More efficient datalog queries: subsumptive tabling beats magic sets.} In Proceedings of the 2011 ACM SIGMOD International Conference on Management of data, SIGMOD 2011.

\bibitem{giraphbook} C. Martella, R. Shaposhnik, D. Logothetis: \textit{Giraph in Action}, Early access edition, http://www.manning.com/martella/.

\bibitem{twitterdata} http://snap.stanford.edu/data/egonets-Twitter.html, Accessed: September 18th, 2014.

\bibitem{aggss} A. Mohapatra, M. Genesereth: \textit{Aggregation in Datalog Under Set Semantics}, http://logic.stanford.edu/reports/LG-2012-01.pdf.

\bibitem{lcagg} M. P. Consens, A. O. Mendelzon: \textit{Low-complexity aggregation in GraphLog and Datalog}, Theoretical Computer Science, v.116 n.1, 1993.

\end{thebibliography}

