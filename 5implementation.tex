

\chapter{Datalog on Spark}\label{r:implementation}

\todo{
  Opis narzędzia, jak się go używa (przykłady), w jaki sposób tłumaczy Dataloga na Sparka, porównania z SociaLite i Spark GraphX API
  + o tym że oryginalny SociaLite nie działa
}


Zalety implementacji na Sparku:
\begin{itemize}
\item dzięki implementacji na Sparku można mieszać zapytania Datalogowe z kodem Sparka, np. wykorzystywać go do rozszerzenia istniejących programów
\item można skorzystać z tego że Spark ma całą integrację z HDFS i innymi
\item uzytkownicy Sparka nie potrzebują dodatkowych inwestycji w  infrastrukturę
\item korzystamy z wbudowanych, dopracowanych i ciągle dopracowywanych optymalizacji Sparka
\item mamy fault-tolerance zapewnione przez Sparka. Czyli dopracowane i plus jest taki, że nie ma konieczności zapisu na dysk: pamiętane są ścieżki wywołań które doprowadziły do określonego wyniku (sprawdzić jak to dokładnie działa)
\end{itemize}


\pomysl{typowe problemy na grafach
porównamy się z SociaLite i Spark Pregel API/GraphX
further work: narzędzie pozwalające sprawdzić lub teoretyczny wynik jak sprawdzać, czy jest monotoniczność pozwalająca robić rekurencyjną agregację
}


