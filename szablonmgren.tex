%
% Niniejszy plik stanowi przyklad formatowania pracy magisterskiej na
% Wydziale MIM UW.  Szkielet uzytych polecen mozna wykorzystywac do
% woli, np. formatujac wlasna prace.
%
% Zawartosc merytoryczna stanowi oryginalnosiagniecie
% naukowosciowe Marcina Wolinskiego.  Wszelkie prawa zastrzezone.

% Copyright (c) 2001 by Marcin Wolinski <M.Wolinski@gust.org.pl>
% Poprawki spowodowane zmianami przepisów - Marcin Szczuka, 1.10.2004
% Poprawki spowodowane zmianami przepisow i ujednolicenie 
% - Seweryn Karlowicz, 05.05.2006
% Zmiany dla wersji angielskiej - Michal Derezinski, 10.05.2012
% dodaj opcje [licencjacka] dla pracy licencjackiej
\documentclass{pracamgren}
%\documentclass[licencjacka]{pracamgren}

%Jesli uzywasz kodowania polskich znakow ISO-8859-2 nastepna linia powinna byc 
%odkomentowana
%\usepackage[latin2]{inputenc}
\usepackage[T1]{fontenc}
%Jesli uzywasz kodowania polskich znakow CP-1250 to ta linia powinna byc 
%odkomentowana
%\usepackage[cp1250]{inputenc}

% Dane magistranta:

\author{Jan Kowalski}

\nralbumu{123456}

\title{Title in English}

\tytul{Tytu{\l} po Polsku}


% kierunek: Mathematics, Informatics, ...
\kierunek{Informatics}

% informatyka - nie okreslamy zakresu (opcja zakomentowana)
% matematyka - zakres moze pozostac nieokreslony,
% a jesli ma byc okreslony dla pracy mgr,
% to przyjmuje jedna z wartosci:
% {metod matematycznych w finansach}
% {metod matematycznych w ubezpieczeniach}
% {matematyki stosowanej}
% {nauczania matematyki}
% Dla pracy licencjackiej mamy natomiast
% mozliwosc wpisania takiej wartosci zakresu:
% {Jednoczesnych Studiow Ekonomiczno--Matematycznych}

% W wersji angielskiej, zakres ma byc po angielsku.
% \zakres{Tu wpisac, jesli trzeba, jedna z opcji podanych wyzej}

% Praca wykonana pod kierunkiem:
% (podac tytul/stopien imie i nazwisko opiekuna
% Instytut
% ew. Wydzial ew. Uczelnia (jezeli nie MIM UW))
% W wersji angielskiej, ma byc po angielsku.
\opiekun{dr Ksawery Iksinski\\
Institute of Informatics\\
}

% miesiac i rok (po angielsku):
\date{May 2012}

%Podac dziedzine wg klasyfikacji Socrates-Erasmus:
\dziedzina{ 
%11.0 Matematyka, Informatyka:\\ 
%11.1 Matematyka\\ 
%11.2 Statystyka\\ 
%11.3 Informatyka\\ 
11.4 Sztuczna inteligencja\\ 
%11.5 Nauki aktuarialne\\
%11.9 Inne nauki matematyczne i informatyczne
}

\dziedzinaang{
%11.0 Mathematics, Informatics:\\ 
%11.1 Mathematics\\ 
%11.2 Statistics\\ 
%11.3 Informatics, Computer Science\\ 
11.4 Artificial Intelligence\\ 
%11.5 Actuarial Science\\
%11.9 Others Mathematics, Informatics
}

%Klasyfikacja tematyczna wedlug AMS (matematyka) lub ACM (informatyka)
\klasyfikacja{I. Metodologie Obliczeniowe\\
  I.2. Sztuczna Inteligencja\\
  I.2.6. Uczenie}

\klasyfikacjaang{I. Computing Methodologies\\
  I.2. Artificial Intelligence\\
  I.2.6. Learning}


% Slowa kluczowe:
\keywords{Szablon, Praca Magisterska, LaTeX}

\keywordsang{Template, Master's Thesis, LaTeX}

% Tu jest dobre miejsce na Twoje wlasne makra i~srodowiska:
\newtheorem{defi}{Definition}[section]

% koniec definicji

\begin{document}
\maketitle


\streszczenie{
To jest streszczenie po polsku, czyli przet{\l}umaczony abstrakt.
To jest streszczenie po polsku, czyli przet{\l}umaczony abstrakt.
To jest streszczenie po polsku, czyli przet{\l}umaczony abstrakt.
To jest streszczenie po polsku, czyli przet{\l}umaczony abstrakt.
To jest streszczenie po polsku, czyli przet{\l}umaczony abstrakt.
To jest streszczenie po polsku, czyli przet{\l}umaczony abstrakt.
To jest streszczenie po polsku, czyli przet{\l}umaczony abstrakt.
To jest streszczenie po polsku, czyli przet{\l}umaczony abstrakt.
To jest streszczenie po polsku, czyli przet{\l}umaczony abstrakt.
To jest streszczenie po polsku, czyli przet{\l}umaczony abstrakt.
To jest streszczenie po polsku, czyli przet{\l}umaczony abstrakt.
}

%tu idzie abstrakt po angielsku na strone poczatkowa
\begin{abstract}
This is the English version of the abstract of the thesis.
This is the English version of the abstract of the thesis.
This is the English version of the abstract of the thesis.
This is the English version of the abstract of the thesis.
This is the English version of the abstract of the thesis.
This is the English version of the abstract of the thesis.
This is the English version of the abstract of the thesis.
This is the English version of the abstract of the thesis.
This is the English version of the abstract of the thesis.
This is the English version of the abstract of the thesis.
This is the English version of the abstract of the thesis.

\end{abstract}

\tableofcontents
%\listoffigures
%\listoftables

\chapter*{Introduction}
\addcontentsline{toc}{chapter}{Introduction}

This is the introduction.

\chapter{First Chapter}

This is the first chapter.

\begin{thebibliography}{99}
\addcontentsline{toc}{chapter}{Bibliography}
\bibitem{one}
Ksawery Iksinski, Some Article, Some Journal, 2000.
\end{thebibliography}

\end{document}